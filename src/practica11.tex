\documentclass{beamer}
\usepackage[spanish]{babel}
\usepackage[utf8]{inputenc}
\usepackage{graphics}
\institute{Facultad de Matemáticas}

\title[Práctica 11]{\huge{Aproximación del número Pi}}
\author[Adriana calvo]{\LARGE{Adriana Calvo}}
\date{\today}
\usetheme{Madrid}


\definecolor{mivioleta}{RGB}{122,59,122}
\definecolor{miazul}{RGB}{0,88,147}
\definecolor{migris}{RGB}{56,61,66}
\setbeamercolor*{palette primary}{use=structure,fg=white,bg=mivioleta}
\setbeamercolor*{palette secundary}{use=structure,fg=white,bg=miazul}
\setbeamercolor*{pallete tertiary}{use=structure,fg=white,bg=migris}



\begin{document}


\begin{frame}
\titlepage
\end{frame}

\begin{frame}
\frametitle{Indice}
\tableofcontents[pausesections]
\end{frame}
\begin{frame}

\end{frame}
\section{Introducción}
\subsection{Definición}
\begin{frame}
\frametitle{Introducción}

El valor de $pi$ se ha obtenido con diversas aproximaciones a lo largo de la historia, siendo una de las constantes matemáticas que más aparece en las ecuaciones de la física, junto con el número $e$. Cabe destacar que el cociente entre la longitud de cualquier circunferencia y la de su diámetro no es constante en geometrías no euclídeas.
El valor numérico de $\pi$, truncado a sus primeras cifras, es el siguiente:\\


\centerline{$\pi \approx 3,14159265358979323846$ }
\end{frame}
\subsection{Motivación y objetos}
\begin{frame}
\frametitle{Motivación y objetivos}
En esta presentación se estudiará las aproximaciones al famoso número $\pi$ así como también el estudio de este por las diferentes culturas.
El número $\pi$ se puede apoximar numéricamente con la siguiente formula:
\ \\


\centerline{$\pi\approx\frac{1}{n}\sum_{i=1}^{n}$$f(x_i)$}




\centerline{donde $f(x)=\frac{4}{1+x^2}$, $x_i=\frac{i-\frac{1}{2}}{n}$ para i=1..n}
\ \\
\end{frame}
\section{Historia del número Pi}
\subsection{Antiguo Egipto}
\begin{frame}
\frametitle{Historia del número Pi}

El valor aproximado de $\pi$ en las antiguas culturas se remonta a la época del escriba egipcio Ahmes en el año 1800 a. C., descrito en el papiro Rhind,3 donde se emplea un valor aproximado de $\pi$ afirmando que el área de un círculo es similar a la de un cuadrado cuyo lado es igual al diámetro del círculo disminuido en 1/9; es decir, igual a 8/9 del diámetro. 
\end{frame}
\begin{frame}
\frametitle{Historia del número Pi}
Entre los ocho documentos matemáticos hallados de la antigua cultura egipcia, en dos se habla de círculos. Uno es el papiro Rhind y el otro es el papiro de Moscú. Sólo en el primero se habla del valor aproximado del número $\pi$. El investigador Otto Neugebauer, en un anexo de su libro The Exact Sciences in Antiquity,4 describe un método inspirado en los problemas del papiro de Ahmes para averiguar el valor de $\pi$, mediante la aproximación del área de un cuadrado de lado 8, a la de un círculo de diámetro 8.

\end{frame}
\subsection{Mesopotámia}
\begin{frame}
\frametitle{Historia del número Pi}
Algunos matemáticos mesopotámicos empleaban, en el cálculo de segmentos, valores de $\pi$ igual a 3, alcanzando en algunos casos valores más aproximados, como el de:\\


\centerline{$\pi$ $\approx3+\frac{1}{8} = 3,125$}
 
\end{frame}
\section{Bibliografía}
\begin{frame}
\frametitle{Bibliografía}
\begin{itemize}

\item{Wikipedia}\\
\item{Apuntes de Matemática Discreta}\\
\item{Web de la Real sociedad Matemática}
\end{itemize}


\end{frame}

\end{document}
